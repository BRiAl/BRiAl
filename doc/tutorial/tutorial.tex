%
%  untitled
%
%  Created by Michael Brickenstein on 2007-11-08.
%  Copyright (c) 2007 Mathematisches Forschungsinstitut Oberwolfach. All rights reserved.
%
\documentclass[]{article}

% Use utf-8 encoding for foreign characters
\usepackage[utf8]{inputenc}

% Setup for fullpage use
\usepackage{fullpage}

% Uncomment some of the following if you use the features
%
% Running Headers and footers
%\usepackage{fancyhdr}

% Multipart figures
%\usepackage{subfigure}

% More symbols
%\usepackage{amsmath}
%\usepackage{amssymb}
%\usepackage{latexsym}

% Surround parts of graphics with box
\usepackage{boxedminipage}

% Package for including code in the document
\usepackage{listings}

% If you want to generate a toc for each chapter (use with book)
\usepackage{minitoc}
\usepackage{xspace}
% This is now the recommended way for checking for PDFLaTeX:
\usepackage{ifpdf}
\ifpdf
\usepackage[pdftex]{graphicx}
\else
\usepackage{graphicx}
\fi
\newenvironment{code}{\begin{verbatim}}{\end{verbatim}}
\newcommand{\PolyBoRi}{{\sc PolyBoRi}\xspace}
\newcommand{\CUDD}{{CUDD}\xspace}
\newcommand{\ite}{{ite}\xspace}
\providecommand{\boolemult}{\ensuremath{{\star\hspace{-.15em}_2\hspace{.05em}}}\xspace}
%\newif\ifpdf
%\ifx\pdfoutput\undefined
%\pdffalse % we are not running PDFLaTeX
%\else
%\pdfoutput=1 % we are running PDFLaTeX
%\pdftrue
%\fi


\title{\PolyBoRi tutorial}
\author{  }




\date{2007-11-08}

\begin{document}

\ifpdf
\DeclareGraphicsExtensions{.pdf, .jpg, .tif}
\else
\DeclareGraphicsExtensions{.eps, .jpg}
\fi

\maketitle


\begin{abstract}
\end{abstract}

\section{Introduction}
\section{How to program efficiently}
The goal of this section is to explain how to get most performance out of \PolyBoRi using the underlying ZDD structure.
This awareness can be seen on several levels
\begin{itemize}
    \item ZDD unaware, pure algebraic programming 
    \item low level friendly programming
    \item replacing algebraic operations by (a composition of) set operations
    \item Cudd recursive programming without caching
    \item Cudd recursive programming with caching
    \item using ZDDs for many other things than polynomial arithmetik
\end{itemize}
\subsection{Low level friendly programming}
\PolyBoRi is implemented as layer over a decision diagram library (\CUDD at the moment).

In \CUDD every diagram node is unique: If two diagrams have the same structure, the are in fact identical (same position in memory).
Another observation is, that \CUDD tries to build a functional style API in the C programming language. This means, that no data is manipulated, only new nodes are created.
Functional programming is in principal very suited for caching and multithreading (at the moment however threading is not possible).
The \ite-operator is the most central function in CUDD. It takes two nodes/diagrams $t$, $e$ and an index $i$ and creates a diagram with root variable $i$ and
then-branch $t$, else-branch $e$. It is necessary that the branches have root variables with bigger index (or are constant).
It creates either exactly one node, or retrieves the correct node from the cache.
Function calls which come essentially down to a single \ite call are very cheap.
For example taking the product $x_1 \boolemult (x_2\boolemult(x3\boolemult (x_4\boolemult x_5)))$ is much cheaper than $((((x_1 \boolemult x_2)\boolemult x3)\boolemult x_4)\boolemult x_5)$.
In the first case, in each step a single not is prepended to the diagram, while in the second case, a completely new diagram is created.
The same argumemt would apply for the addition of these variables.
This example shows, that having a little bit background about the data structure, it is often possible to write code, that looks as well algebraic as provides good performance.
\bibliographystyle{plain}

\bibliography{}
\end{document}
