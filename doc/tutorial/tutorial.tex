%
%  untitled
%
%  Created by Michael Brickenstein on 2007-11-08.
%  Copyright (c) 2007 Mathematisches Forschungsinstitut Oberwolfach. All rights reserved.
%
\documentclass[]{article}

% Use utf-8 encoding for foreign characters
\usepackage[utf8]{inputenc}

% Setup for fullpage use
\usepackage{fullpage}
\usepackage{amssymb}
% Uncomment some of the following if you use the features
%
% Running Headers and footers
%\usepackage{fancyhdr}

% Multipart figures
%\usepackage{subfigure}

% More symbols
%\usepackage{amsmath}
%\usepackage{amssymb}
%\usepackage{latexsym}

% Surround parts of graphics with box
\usepackage{boxedminipage}

% Package for including code in the document
\usepackage{listings}

% If you want to generate a toc for each chapter (use with book)
\usepackage{minitoc}
\usepackage{xspace}
% This is now the recommended way for checking for PDFLaTeX:
\usepackage{ifpdf}
\ifpdf
\usepackage[pdftex]{graphicx}
\else
\usepackage{graphicx}
\fi
\newenvironment{code}{\begin{verbatim}}{\end{verbatim}}
\newcommand{\PolyBoRi}{{\sc PolyBoRi}\xspace}
\newcommand{\Groebner}{Gröbner\xspace}
\newcommand{\CUDD}{{CUDD}\xspace}
\newcommand{\ite}{{ite}\xspace}
\newcommand{\functionname}[1]{\textit{#1}\xspace}
\newcommand{\pythonconstant}[1]{\textit{#1}\xspace}
\newcommand{\explfieldequations}{{x_1^2+x_1,\ldots,x_n^2+x_n}}
\providecommand{\boolemult}{\ensuremath{{\star\hspace{-.15em}_2\hspace{.05em}}}\xspace}
\newcounter{thm}
\newtheorem{definition}[thm]{Definition}

\newcommand{\Ztwo}{\mathbb{Z}_2}
%\newif\ifpdf
%\ifx\pdfoutput\undefined
%\pdffalse % we are not running PDFLaTeX
%\else
%\pdfoutput=1 % we are running PDFLaTeX
%\pdftrue
%\fi


\title{\PolyBoRi Tutorial}
\author{Michael Brickenstein}




\date{\today}

\begin{document}

\ifpdf
\DeclareGraphicsExtensions{.pdf, .jpg, .tif}
\else
\DeclareGraphicsExtensions{.eps, .jpg}
\fi

\maketitle
\newcommand{\Singular}{{\sc Singular}}



\section{Introduction}
\subsection{Interfaces}
The core of \PolyBoRi is a C++ library. On top of it there exists a Python interface.
Additionally to the Python interface a integration into SAGE was provided by Burcin Erocal.
The main difference is, that \PolyBoRi's built-in Python interface makes use of
the boost library, while the SAGE interface relies on Cython. 
However the wrappers for SAGE and the original Python interface are designed, that it is possible to run the same code under both bindings.

We provide an interactive shell for \PolyBoRi using ipython for the SAGE
interface~(which is invoked the command~\texttt{sage}) as well as for the
built-in one, which can be accessed by typing~\texttt{ipbori}
at the command line prompt.


In ipbori a global ring is predefined and a set of variables called  $x(0), \ldots, x(9999)$. The default ordering is lexicographical ordering (lp).
\subsection{Ordering}
The monomial ordering can be changed by calling
\functionname{change\_ordering}(\functionname{Ordering.code}), where \functionname{code} can be either \functionname{lp} (lexicographical ordering), \functionname{dlex} (degree lexicographical ordering), \functionname{dp\_asc} (degree reverse lexicographical ordering with ascending variable ordering), \functionname{block\_dlex} or \functionname{block\_dp\_asc} (for ordering composed out of blocks in the corresponding ordering). When using block ordering, after changing to that ordering, blocks have to be defined using the \functionname{append\_ring\_block} function.

In contrast to the lexicographical, degree lexicographical ordering, and the degree reverse lexicographical ordering in \Singular, our degree reverse lexicographical ordering has a reverse variable order (the first ring variable is smaller than the second, the second smaller than the third). This is a result of the fact, that efficient implementation of monomial orderings using ZDD structures is quite difficult (and the performance depends on the ordering).
\paragraph{Example: Degree reverse lexicographical ordering}
\begin{verbatim}
r=Ring(1000)
change_ordering(dp_asc)
\end{verbatim}

\paragraph{Example: Block orderings}
\begin{verbatim}
r=Ring(1000)
change_ordering(block_dp_asc)
append_ring_block(10)
append_ring_block(20)
\end{verbatim}
In this example, we have an ordering composed of three blocks, the first with ten variables, the second contains the $x(10), \ldots x(19)$ (per default indices start at 0).
\subsection{Arithmetic}
Basic arithmetic is provided in the domain of Boolean polynomials. Boolean Polynomial polynomials are polynomials over $\Ztwo$ where the maximal degree per variable is one.
If exponents bigger than one per variable appear reduction by the field ideal (polynomials of the form $x^2+x$) is done automatically.
\begin{verbatim}
In [1]: Polynomial(1)+Polynomial(1)
Out[1]: 0

In [2]: x(1)*x(1)
Out[2]: x(1)

In [3]: (x(1)+x(2))*(x(1)+x(3))
Out[3]: x(1)*x(2) + x(1)*x(3) + x(1) + x(2)*x(3)
\end{verbatim}

\subsection{Set operations}
In addition to polynomials  \PolyBoRi implements a data type for sets of monomials, called \functionname{BooleSet}.
This data type is also implemented on the top of ZDDs and allows to see
polynomials
from a different angle. Also, it makes high-level set operations possible, which are in most cases faster than operations handling individual terms, because the complexity of the algorithms depends only on the structure of the diagrams.

Polynomials can easily be converted to \functionname{BooleSet}s by using the
member function \verb|set()|.
\begin{verbatim}
In [1]: f=x(2)*x(3)+x(1)*x(3)+x(2)+x(4)

In [2]: f
Out[2]: x(1)*x(3) + x(2)*x(3) + x(2) + x(4)

In [3]: f.set()
Out[3]: {{x(1),x(3)}, {x(2),x(3)}, {x(2)}, {x(4)}}
\end{verbatim}
%
One of the most common operations is to split the set into cofactors of a
variable. This illustrates the following example.
%
\begin{verbatim}
In [4]: s0=f.set().subset0(x(2).index())

In [5]: s0
Out[5]: {{x(1),x(3)}, {x(4)}}

In [6]: s1=f.set().subset1(x(2).index())

In [7]: s1
Out[7]: {{x(3)}, {}}

In [8]: f==Polynomial(s1)*x(2)+Polynomial(s0)
Out[8]: True
\end{verbatim}
%
But also operations on higher levels are possible, like the calculation of the minimal terms (with respect to division) in a~\functionname{BooleSet}:
\begin{verbatim}
In [1]: f=x(2)*x(3)+x(1)*x(3)+x(2)+x(4)

In [2]: f.set()
Out[2]: {{x(1),x(3)}, {x(2),x(3)}, {x(2)}, {x(4)}}

In [3]: f.set().minimalElements()
Out[3]: {{x(1),x(3)}, {x(2)}, {x(4)}}
\end{verbatim}
\subsection{\Groebner bases}
\Groebner bases functionality is available using the function \functionname{groebner\_basis} from polybori.gbcore.
It has quite a lot of options and a exchangable heuristic.
In principle, there exist  standard settings, but~-- in case an option is not
provided explicitely by the user~-- the active heuristic function
may decide dynamically by taking the ideal, the ordering and the other options into account, which is the best configuration.
\begin{verbatim}
In [1]: groebner_basis([x(1)+x(2),(x(2)+x(1)+1)*x(3)])
Out[1]: [x(1) + x(2), x(3)]
\end{verbatim}

There exists a set of default options for \functionname{groebner\_basis}.
They can be seen, but not manipulated via accessing \verb|groebner_basis.options|.
A second layer of heuristics is incorporated into the \functionname{groebner\_basis}-function, to choose dynamically the best options depending on the ordering and the given ideal.
Every option given explicitely by the user takes effect, but for the other options the default may be overwritten by the script.
This behaviour can be turned off by calling
\begin{verbatim}
groebner_basis(I,heuristic=False).
\end{verbatim}

Important options are the following:
\begin{itemize}
    \item \verb|other_ordering_first|, possible values are \pythonconstant{False} or an ordering code.
    In practice, many Boolean examples have very few solutions and a very easy \Groebner basis. So, a complex walk algorithm (which cannot be implemented using the \PolyBoRi data structures) seems unnecessary, as such \Groebner bases can be converted quite fast by the 
    normal Buchberger algorithm from one ordering into another ordering.
    \item \verb|faugere|, turn off or on the linear algebra
    \item \verb|linear_algebra_in_last_block|, this affects the last block of block orderings and degree orderings. If it is set to \pythonconstant{True} linear algebra takes affect in this block.
    \item \verb|selection_size|, maximum number of polynomials for parallel reductions
    \item \verb|prot|, turn off or on the protocol
\end{itemize}
\section{How to program efficiently}
The goal of this section is to explain how to get most performance out of \PolyBoRi using the underlying ZDD structure.
This awareness can be seen on several levels
\begin{itemize}
    \item ZDD unaware, pure algebraic programming 
    \item low level friendly programming
    \item replacing algebraic operations by (a composition of) set operations
    \item decision-diagram style recursive programming without caching
    \item decision-diagram style recursive programming with caching
    \item using ZDDs for many other things than polynomial arithmetics
\end{itemize}
\subsection{Low level friendly programming}
\label{low-level-friendly}
\PolyBoRi is implemented as layer over a decision diagram library (\CUDD at the moment).

In \CUDD every diagram node is unique: If two diagrams have the same structure, they are in fact identical (same position in memory).
Another observation is, that \CUDD tries to build a functional style API in the C programming language. This means, that no data is manipulated, only new nodes are created.
Functional programming is a priori very suited for caching and multithreading (at the moment however threading is not possible in \PolyBoRi).
The \ite-operator is the most central function in CUDD. It takes two nodes/diagrams $t$, $e$ and an index $i$ and creates a diagram with root variable $i$ and
then-branch $t$, else-branch $e$. It is necessary that the branches have root variables with bigger index (or are constant).
It creates either exactly one node, or retrieves the correct node from the cache.
Function calls which come essentially down to a single \ite call are very cheap.

For example taking the product $x_1 \boolemult (x_2\boolemult(x3\boolemult (x_4\boolemult x_5)))$ is much cheaper than $((((x_1 \boolemult x_2)\boolemult x3)\boolemult x_4)\boolemult x_5)$.
In the first case, in each step a single not is prepended to the diagram, while in the second case, a completely new diagram is created.
The same argument would apply for the addition of these variables.
This example shows, that having a little bit background about the data structure, it is often possible to write code, that looks as well algebraic as provides good performance.

\subsection{Replace algebra by set operations}
Often there is an alternative description in terms of set operations for algebraic operations, which is much faster.

\subsubsection{Construct power sets}
An example for this behaviour is the calculation of power sets (sets of monomials/polynomials containing each term in the specified variables).
The following code constructs such a power set very inefficiently for the first three variables:
\begin{verbatim}
sum([x(1)**i1*x(2)**i2*x(3)**i3 for i1 in (0,1) for i2 in (0,1) for i3 in (0,1)])
\end{verbatim}
The algorithm has of course exponential complexity in the number of variables.
The resulting ZDD however has only as many nodes as variables.
In fact it can be constructed directly using the following function (from specialsets.py).
\begin{verbatim}
def power_set(variable_indices):
    variable_indices=list(reversed(list(set(variable_indices))))
    res=Polynomial(1).set()
    for v in variable_indices:
        res=if_then_else(v,res,res)
    return res
\end{verbatim}
Note, that we switched from polynomials to Boolean sets. We inverse the order of variable indices for iteration to make the computation compatible with the principles in \ref{low-level-friendly} (simple \ite operators instead of complex operations in each step).
\subsubsection{All monomials of degree d}
\begin{verbatim}
def all_monomials_of_degree_d(d,variable_indices):
    if d == 0:
        return Polynomial(1).set()
    if len(variable_indices) == 0:
        return BooleSet()
    variable_indices = list(reversed(list(set(variable_indices))))
    m = Variable(variable_indices[-1])
    for v in variable_indices[:-1]:
        m = Variable(v) + m
    m = m.set()
    def do_all_monomials(d):
        if d == 0:
            return Polynomial(1).set()
        else:
            prev = do_all_monomials(d-1)
            return prev.cartesianProduct(m).diff(prev)
    return do_all_monomials(d)
\end{verbatim}
We use the set of all monomials of one degree lower using the cartesian product with the set of variables and remove every term, where the degree did not increase (boolean multiplication: $x^2=x$).

\section{Case study: Graded part of a polynomial}
In the following we will show five variants to implement a function, that computes the sum of all terms of degree $d$ in a polynomial $f$.
\subsection{Simple, algebraic solution}
\begin{verbatim}
def simple_graded(f, d):
    return sum((t for t in f if t.deg()==d))   
\end{verbatim}
This solution is obvious, but quite slow.
\subsection{Low level friendly, algebraic solution}
\begin{verbatim}
def friendly_graded(f, d):
    vec=BoolePolynomialVector()
    for t in f:
        if t.deg()!=d:
            continue
        else:
            vec.append(t)
    return add_up_polynomials(vec)
\end{verbatim}
We leave it to the heuristic of the \functionname{add\_up\_polynomials} function how to add up the monomials. For example a divide and conquer strategy is quite good here.
\subsection{Highlevel with set operations}
\begin{verbatim}
def highlevel_graded(f,d):
    return Polynomial(f.set().intersect(all_monomials_of_degree_d(d,f.vars())))
\end{verbatim}
This solution build on the fast intersection algorithm and decomposes the task in just two set operations, which is very good.

However it can be quite inefficient, when f has many variables.
This can increase the number of steps in the intersection algorithm (which takes with high probability the else branch of the second argument in each step).
\subsection{Recursive}
The repeated unnecessary iteration over all variables in $f$ (during the \functionname{intersection} call in the last section) can be avoided by taking just integers as second argument for the recursive algorithm (in the last section this was \functionname{intersection}).

\begin{verbatim}
def recursive_graded(f,d):
    def do_recursive_graded(f,d):
        if f.empty():
            return f
        if d==0:
            if Monomial() in f:
                return Polynomial(1).set()
            else:
                return BooleSet()
        else:
            nav=f.navigation()
            if nav.constant():
                return BooleSet()
            return if_then_else(
                nav.value(),
                do_recursive_graded(BooleSet(nav.thenBranch()),d-1),
                do_recursive_graded(BooleSet(nav.elseBranch()),d))
    return Polynomial(do_recursive_graded(f.set(),d))
        
\end{verbatim}
Recursive implementations are very compatible with our data structures, so are quite fast. However this implementation does not use any caching techniques. Cudd recursive caching requires functions to have one, two or three parameters, which are of ZDD structure (so no integers).
Of course we can encode the degree $d$ by the d-th Variable in the Polynomial
ring.

\subsection{Decision-diagram style recursive implementation in \PolyBoRi}
The C++ implementation of the functionality in \PolyBoRi is given in this section, which is recursive and uses caching techniques.
\begin{verbatim}
// determine the part of a polynomials of a given degree
template <class CacheType, class NaviType, class DegType, class SetType>
SetType
dd_graded_part(const CacheType& cache, NaviType navi, DegType deg,  
               SetType init) {


  if (deg == 0) {
    while(!navi.isConstant())
      navi.incrementElse();
    return SetType(navi);
  }

  if(navi.isConstant())
    return SetType();

  // Look whether result was cached before
  NaviType cached = cache.find(navi, deg);

  if (cached.isValid())
    return SetType(cached);

  SetType result = 
    SetType(*navi,  
            dd_graded_part(cache, navi.thenBranch(), deg - 1, init),
            dd_graded_part(cache, navi.elseBranch(), deg, init)
            );

  // store result for later reuse
  cache.insert(navi, deg, result.navigation());

  return result;
}
\end{verbatim}
The encoding of integers for the degree as variable is done implicitely by our cache lookup functions.

\section{Case study: Evaluation of a polynomial}

\subsection{Substitute a single variable $x$ in a polynomial by a constant $c$}

Given a Boolean polynomial $f$, a variable $x$ and a constant $c$, we want to plug in the constant $c$ for the variable $x$.
\subsubsection{Naive approach}
The following code shows how to tackle the problem, by manipulating individual terms.
While this is a very direct approach, it is quite slow.
The method \functionname{reducibleBy} gives a test for divisibility.
\begin{verbatim}
def subst(f,x,c):
    if c==1:
        return sum([t/x for t in f if t.reducibleBy(x)])+\
            sum([t for t in f if not t.reducibleBy(x)])
    else:
        #c==0
        return sum([t for t in f if not t.reducibleBy(x)])

\end{verbatim}
\subsubsection{Solution 1: Set operations}
In fact, the problem can be tackled quite efficiently using set operations.
\begin{verbatim}
def subst(f,x,c):
   i=x.index()
   c=Polynomial(c)#if c was int is now converted mod 2, so comparison to int(0) makes sense
   s=f.set()
   if c==0:
       #terms with x evaluate to zero
       return Polynomial(s.subset0(i))
   else:
       #c==1
       return Polynomial(s.subset1(i))+Polynomial(s.subset0(i))    
\end{verbatim}
\subsubsection{Solution 2: Linear Lead rewriting systems}
On the other hand, this linear rewriting forms a rewriting problem and can be solved by calculating a normal form against a Gröbner basis.
In this case the system is $\{x+c\} \cup \{\explfieldequations\}$ (we assume that $x=x_i$ for some $i$).
For this special case, that all Boolean polynomials have pairwise different linear leading terms,
there exist special functions.

First, we encode the system $\{x+c\}$ into one diagram
\begin{verbatim}
d=ll_encode([x+c])    
\end{verbatim}
%
This is a special format representing a set of such polynomials in one diagram, which is used by several procedures in
\PolyBoRi.
Then we may
reduce~$f$ by this rewriting system
\begin{verbatim}
ll_red_nf_noredsb(f,d)  
\end{verbatim}
%
%
This can be simplified in our special case in two ways.
\begin{enumerate}
    \item If our system consists of exactly \textbf{one} Boolean polynomial,
    \verb|ll_encode| does essentially  a type conversion only (and much overhead).
    This type conversion can be done implicitely (at least using the
\verb|boost::python|-based  interface \verb|ipbori|).

    So you may call
\begin{verbatim}
ll_red_nf_noredsb(f,x+c)  
\end{verbatim}
%
    In this case, there is no need for calling \verb|ll_encode|.
    The second argument is converted implicitely to BooleSet.
    \item A second optimization is to call just
\begin{verbatim}
ll_red_nf(f,x+c)
\end{verbatim}
    %
    As $\{x+c\}$ is a reduced Boolean Gröbner basis (equivalently $\{x+c,\explfieldequations\}\backslash \{x^2+x\}$ is a reduced Gröbner basis).
\end{enumerate}



\subsection{Evaluate a polynomial by plugging in a constant for each variable}
    We want to a polynomial
    $f(x_1,\ldots, x_n)$
    by
    $x_i\mapsto c_i$, where
    $c_1,\ldots, c_y$ are constants.
\subsubsection{Naive approach}
First, we show it in a naive way, similar to the first solution above.

\begin{verbatim}
def evaluate(f,m):
    res=0
    for term in f:
        product=1
        for index in term:
            product=m[index]*product
        res=res+product
    return Polynomial(res)
\end{verbatim}


\subsubsection{Solution 1: $n$ set operations}
The following approach is faster, as it does not involve individual terms, but set operations

\begin{verbatim}
def evaluate(f,m):
   while not f.constant():
       nav=f.navigation()
       i=nav.value()
       c=m[i]
       if c==0:
           #terms with x evaluate to zero
           f=Polynomial(nav.thenBranch())
       else:
           #c==1
           f=Polynomial(nav.thenBranch())+Polynomial(nav.elseBranch())
       return f   
\end{verbatim}
For example, the call
\begin{verbatim}
evaluate(x(1)+x(2),{x(1).index():1,x(2).index():0})  
\end{verbatim}
results in~\verb|1|.



We deal here with navigators, which is dangerous, because
they do not increase the internal reference count on the represented polynomial
substructure. So, one has
to ensure, that~$f$ is still valid, as long as we use a navigator on~$f$.
But it will show its value on optimized code (e.\,g.\ PyRex), where it causes
less overhead. 
A second point, why it is desirable to use navigators is, that their
\verb|thenBranch|- and \verb|elseBranch|-methods immediately return~(without
further calculations) the
results of the \verb|subset0| and \verb|subset1|-functions, when the latter are
called together  with the top variable of the diagram~$f$.
%
In this example, this is the crucial point in terms of performance.
But, since we already call the polynomial construction on the branches,
reference counting of the corresponding subpolynomials is done anyway.

This is quite fast, but suboptimal, because only the inner functions (additions) use caching.
%
Furthermore, it contradicts the usual ZDD recursion and generates complex intermediate results.

\subsubsection{Solution 2: Linear Lead rewriting systems}
The same problem can also be tackled by the linear-lead routines. In the case, when
all variables are substituted by  constants, all intermediate results
(generated during \verb|ll_red_nf|/\verb|ll_red_nf_noredsb|) are constant.
In general, we consider the overhead of generating the encoding $d$ as small, 
since it consists of very few, tiny ZDD operations only (and some Python overhead in the quite general \verb|ll_encode|).
\begin{verbatim}
d=ll_encode([x+cx,y+cy])
ll_red_nf_noredsb(f,d)
\end{verbatim}
%
%
Since the tails of the polynomials in the rewriting system   consist of
constants only, this forms also a
reduced Gröbner basis. Therefore, you may just call
\begin{verbatim}
ll_red_nf(f,d)   
\end{verbatim}
%
It is assumed, that this is the fasted way.
%


\subsection{Generalization: Linear Lexicographical Lead Rewriting Systems}

We used \verb|ll_red_nf|/\verb|ll_red_nf_noredsb| functions on rewriting systems, where the tails of the polynomials was constant and the leading term linear.
They can be used in a more general setting (which allows to eliminate auxiliary variable).
\begin{definition}
Let $L$ be a list of Boolean polynomials.
If all element $p$ of $L$ have pairwise different leading terms with respect to lexicographical ordering.
Then we call $L$ a \textbf{linear lexicographical lead rewriting system}.
\end{definition}
We know that such a system forms together with the complete set of field equations \Groebner basis w.\,r.\,t. lexicographical ordering.

In particular we can use \verb|ll_red_nf| to speedup substitution of a variable $x$ by a value $v$ also in the more general case, that the lexicographical leading term of $x+v$ is equal to $x$.
This can be tested most efficiently by the expression
\begin{verbatim}
x.set().navigation().value()>v.set().navigation().value().
\end{verbatim}

In many cases, we have a bigger equation system, where many variables have a linear leading term w.\,r.\,t. lexicographical ordering (at least one can optimize the formulation of the equations to fulfill these condition).

This systems can be handled by the function \functionname{eliminate} in the module \functionname{polybori.ll}.
I returns three results
\begin{enumerate}
    \item a maximal subset $L$ of the equation system, which forms a linear lexicographical lexicographical rewriting system.
    \item a normal form algorithm $f$, s.\,t. $f(p)$ forms a reduced normal form of $p$ against the \Groebner basis consisting of $L$ and the field equations.
    \item a list of polynomials $R$, which are in reduced normal form against $L$, s.\,t. $L\cup R$ spans modulo field equations the same ideal as the original equation system.
\end{enumerate}

\begin{verbatim}
In [1]: from polybori.ll import eliminate

In [2]: E=[x(1)+1,x(1)+x(2),x(2)+x(3)*x(4)]

In [3]: E=[x(1)+1,x(1)+x(2),x(2)+x(3)*x(4)]
KeyboardInterrupt

In [3]: (L,f,R)=eliminate(E)

In [4]: L
Out[4]: [x(1) + 1, x(2) + x(3)*x(4)]

In [5]: R
Out[5]: [x(3)*x(4) + 1]

In [6]: f(x(1)+x(2))
Out[6]: x(3)*x(4) + 1
\end{verbatim}
\bibliographystyle{plain}

% As long as there are noe citations...
%\bibliography{}
\end{document}
